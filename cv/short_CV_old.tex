%%%%%%%%%%%%%%%%%%%%%%%%%%%%%%%%%%%%%%%%%
% "ModernCV" CV and Cover Letter
% LaTeX Template
% Version 1.11 (19/6/14)
%
% This template has been downloaded from:
% http://www.LaTeXTemplates.com
%
% Original author:
% Xavier Danaux (xdanaux@gmail.com)
%
% License:
% CC BY-NC-SA 3.0 (http://  creativecommons.org/licenses/by-nc-sa/3.0/) 
%
% Important note:
% This template requires the moderncv.cls and .sty files to be in the same 
% directory as this .tex file. These files provide the resume style and themes 
% used for structuring the document.
%
%%%%%%%%%%%%%%%%%%%%%%%%%%%%%%%%%%%%%%%%%

%----------------------------------------------------------------------------------------
%	PACKAGES AND OTHER DOCUMENT CONFIGURATIONS
%----------------------------------------------------------------------------------------

\documentclass[11pt,a4paper,sans]{moderncv} % Font sizes: 10, 11, or 12; paper sizes: a4paper, letterpaper, a5paper, legalpaper, executivepaper or landscape; font families: sans or roman
\usepackage[T1]{fontenc}
\usepackage[utf8]{inputenc}
\usepackage{lmodern}
\usepackage[francais]{layout}
\moderncvstyle{classic} % CV theme - options include: 'casual' (default), 'classic', 'oldstyle' and 'banking'
\moderncvcolor{blue} % CV color - options include: 'blue' (default), 'orange', 'green', 'red', 'purple', 'grey' and 'black'

\usepackage{lipsum} % Used for inserting dummy 'Lorem ipsum' text into the template

\usepackage[scale=0.90]{geometry} % Reduce document margins
\setlength{\hintscolumnwidth}{2.3cm} % Uncomment to change the width of the dates column
\setlength{\makecvtitlenamewidth}{10cm} % For the 'classic' style, uncomment to adjust the width of the space allocated to your name
\setlength{\hoffset}{0.7cm}
\setlength{\textwidth}{6in}
\setlength{\textheight}{10in}
\setlength{\headsep}{25pt}
%----------------------------------------------------------------------------------------
%	NAME AND CONTACT INFORMATION SECTION
%----------------------------------------------------------------------------------------

\firstname{Michael} % Your first name
\familyname{Arbel} % Your last name

% All information in this block is optional, comment out any lines you don't need
\title{Curriculum Vitae }
\address{Centre Inria Grenoble - Rhône-Alpes\\655 Avenue de l'Europe\\}{38330 Montbonnot-Saint-Martin
 \--- France}
\mobile{+33 (0) 6 64 91 16 18}
\email{michael.n.arbel@gmail.com}
\homepage{https://michaelarbel.github.io}{}
%----------------------------------------------------------------------------------------

%%%%%%%%%%%%%%%%%%%%%%%%%%%%%%%%%%%%%%%%%%%%%%%%%%%%%%%%%%%%%%%%%%%%%%%%%%%%%%
%%%%%%%%%%%%%%%%%%%%%%%%%%%%%%%%%%%%%%%%%%%%%%%%%%%%%%%%%%%%%%%%%%%%%%%%%%%%%%
%%%%%% BLACK MAGIC %%%%%%%%%
%%%%%%%%%%%%%%%%%%%%%%%%%%%%%%%%%%%%%%%%%%%%%%%%%%%%%%%%%%%%%%%%%%%%%%%%%%%%%%
%%%%%%%%%%%%%%%%%%%%%%%%%%%%%%%%%%%%%%%%%%%%%%%%%%%%%%%%%%%%%%%%%%%%%%%%%%%%%%

%\usepackage[colorlinks=true,urlcolor=black!50!blue,linkcolor=black,citecolor=black!30!red]{hyperref}

% bibliographie 
\usepackage[defernumbers=true,style=numeric,maxbibnames=9,maxcitenames=1,uniquelist=false,backend=biber,giveninits=true,sorting=none]{biblatex} 
\addbibresource{biblio.bib}
%\addbibresource{biblioapp.bib}
%\addbibresource{bibliows.bib}

%\addbibresource{/Users/MichaelArbel/Documents/Papers/bibliography.bib}
%\usepackage{hyperref}
% Pour trier mes papiers des autres
% \DeclareSourcemap{
%   \maps[datatype=bibtex]{
%     \map{
%       \step[fieldsource=author,
%             match=Rudi,
%             final]
%       \step[fieldset=keywords, fieldvalue=mine]
%     }
%   }   
% } 
    
%Append keywords to identify different bibliography entries.
\DeclareSourcemap{
  \maps[datatype=bibtex, overwrite]{
    \map{
      \perdatasource{biblio.bib}
      %\step[fieldset=keywords, fieldvalue=,append]
      %\step[fieldsource=keywords,notmatch=important,final]
      \step[fieldset=keywords, fieldvalue=mine]
    }
%    \map{
%      \perdatasource{bibliows.bib}
%      \step[fieldset=keywords, fieldvalue=,append]
%      \step[fieldsource=keywords,notmatch=important,final]
%      \step[fieldset=keywords, fieldvalue=minews]
%    }
    \map{
      \perdatasource{biblio.bib}
      %\step[fieldset=keywords, fieldvalue=,append]
      %\step[fieldsource=keywords,notmatch=important,final]
      \step[fieldset=keywords, fieldvalue=minesub]
    }  
%    \map{
%      \perdatasource{biblionat.bib}
%      \step[fieldset=keywords, fieldvalue=,append]
%      \step[fieldsource=keywords,notmatch=important,final]
%      \step[fieldset=keywords, fieldvalue=minenat]
%    } 
    \map{ 
      \perdatasource{biblio.bib}
      \step[fieldset=keywords, fieldvalue=secondary]
    }
%    \map{   
%      \perdatasource{/Users/MichaelArbel/Documents/Papers/bibliography.bib}
%      \step[fieldset=keywords, fieldvalue=secondary]
%    }
    \map{ 
      \perdatasource{biblio.bib}
      \step[fieldset=keywords, fieldvalue=bibcode]
    }
    \map{ 
      \perdatasource{biblio.bib}
      \step[fieldset=keywords, fieldvalue=bibcode_ext]
    }
  }
}

\makeatletter \newrobustcmd*{\mknumAlph}[1]{% 
\begingroup \blx@tempcnta=#1\relax \ifnum\blx@tempcnta>702 % 
\else \ifnum\blx@tempcnta>26 % 
\advance\blx@tempcnta\m@ne \divide\blx@tempcnta26\relax \blx@numAlph\blx@tempcnta \multiply\blx@tempcnta26\relax \blx@tempcnta=\numexpr#1-\blx@tempcnta\relax \fi \fi \blx@numAlph\blx@tempcnta \endgroup}
\def\blx@numAlph#1{% 
\ifcase#1\relax\blx@warning@entry{Value out of range}\number#1\or A\or B\or C\or D\or E\or F\or G\or H\or I\or J\or K\or L\or M\or N\or O\or P\or Q\or R\or S\or T\or U\or V\or W\or X\or Y\or Z\else \blx@warning@entry{Value out of range}\number#1\fi} \makeatother 


\makeatletter
\newrobustcmd*{\mknumcode}[1]{\textcolor{black!70!green}{P#1}}
\makeatother
%
\makeatletter
\newrobustcmd*{\mknumcodeext}[1]{\textcolor{black!70!green}{S#1}}
\makeatother

\makeatletter
\newrobustcmd*{\mknummine}[1]{\textcolor{black!20!blue}{#1}}
\makeatother
 
\DeclareFieldFormat{labelnumber}{\ifkeyword{secondary}{\mknummine{#1}}{\ifkeyword{bibcode}{\mknumcode{#1}}{\ifkeyword{bibcode_ext}{\mknumcodeext{#1}}{\ifkeyword{mine}{\mknumAlph{#1}}{\ifkeyword{minesub}{\mknumAlph{#1}}{#1}}}}}}
%mknumAlph
%\DeclareFieldFormat{labelnumber}{\ifkeyword{bibcode}{\mknumcode{#1}}{#1}}
%
%\DeclareFieldFormat{labelnumber}{\ifkeyword{bibcode_ext}{\mknumcodeext{#1}{#1}} 






% pour éviter les nom en majuscules
\DefineBibliographyExtras{french}{% 
  \renewcommand{\mkbibnamelast}[1]{{\hyphenrules{nohyphenation}#1}}% 
} 

% The following definition is copied from authortitle.bbx/authoryear.bbx
\defbibenvironment{nolabelbib}
  {\list
     {}
     {\setlength{\leftmargin}{\bibhang}%
      \setlength{\itemindent}{-\leftmargin}%
      \setlength{\itemsep}{\bibitemsep}%
      \setlength{\parsep}{\bibparsep}}}
  {\endlist}
  {\item}



%%% 

\newcommand{\pub}[1]{\hfill {\small \normalfont \sffamily\color{blue!50!black} (Related publication: \autocite{#1})}\newline\noindent}
\newcommand{\pubs}[1]{\hfill {\small \normalfont \sffamily\color{blue!50!black} (#1)}\newline\noindent}





%%%%%%%%%%%%%%%%%%%%%%%%%%%%%%%%%%%%%%%%%%%%%%%%%%%%%%%%%%%%%%%%%%%%%%%%%%%%%%%%%%%%%%%%%%%%%%%%%%%%%%%%%%%%%%%%%%%%%%%%%%%%%%%%%%%%%%%%%%%%%%%%%%%%%%%%%%%%
%%%%%%%%%%%%%%%%%%%%%%%%%%%%%%%%%%%%%%%%%%%%%%%%%%%%%%%%%%%%%%%%%%%%%%%%%%%%%%%%%%%%%%%%%%%%%%%%%%%%%%%%%%%%%%%%%%%%%%%%%%%%%%%%%%%%%%%%%%%%%%%%%%%%%%%%%%%%
% END BLACK MAGIC
%%%%%%%%%%%%%%%%%%%%%%%%%%%%%%%%%%%%%%%%%%%%%%%%%%%%%%%%%%%%%%%%%%%%%%%%%%%%%%%%%%%%%%%%%%%%%%%%%%%%%%%%%%%%%%%%%%%%%%%%%%%%%%%%%%%%%%%%%%%%%%%%%%%%%%%%%%%%
%%%%%%%%%%%%%%%%%%%%%%%%%%%%%%%%%%%%%%%%%%%%%%%%%%%%%%%%%%%%%%%%%%%%%%%%%%%%%%%%%%%%%%%%%%%%%%%%%%%%%%%%%%%%%%%%%%%%%%%%%%%%%%%%%%%%%%%%%%%%%%%%%%%%%%%%%%%%

% 


\begin{document}

\makecvtitle % Print the CV title

%----------------------------------------------------------------------------------------
%	EDUCATION SECTION
%----------------------------------------------------------------------------------------

\section{Education}
\cventry{2016-2021}{Gatsby Computational Neuroscience Unit - UCL }{London,  UK}{}{}
{\textit{PhD in Machine Learning\\}
Advisor: Arthur Gretton\\
Subject: Regularization and Optimization of Generative Adversarial Networks.} 
\cventry{2014-2015}{École Normale Supérieure}{Cachan, France}{}{}
{\textit{MSc in Machine Learning and Computer Vision (MVA). High honors.\\}
Courses: Graphical models, Kernel methods, Convex optimization, Reinforcement Learning, Object Recognition and Computer Vision. } % Arguments not required can be left empty
\cventry{2011-2014}{École Polytechnique}{Palaiseau, France}{}{}
{\textit{B.S. \& M.S. in Applied Mathematics with minor physics. GPA: 3.93/4.\\}
Major: Probability and Measure theory, Statistics, Learning theory, Dynamical systems, Distribution theory, Stochastic Calculus.\\ 
Minor: Quantum mechanics, Statistical physics, Special Relativity. 
}
\cventry{2009-2011}{Moulay Youssef \& Omar Ibn Al Khattab}{ Rabat \& Meknes, Morocco}{}
{}{Preparatory classes for the French national entrance exams for admission to the ‘Grandes Ecoles’ Science and Engineering schools.
}

 
%----------------------------------------------------------------------------------------
%	WORK EXPERIENCE SECTION
%----------------------------------------------------------------------------------------
\section{Research and Work Experience}
\cventry{2021-Now}{Starting Research Fellow}{THOTH Team}{INRIA Grenoble Rhône-Alpes}{}{Unsupervised Representation learning, Optimization and Sampling.}
\cventry{2016-2021}{Gatsby Computational Neuroscience Unit - UCL }{London,  UK}{}{}
{Adivsor: Arthur Gretton\\
- Methods for Generative Adversarial methods \\
- Optimization methods based on Optimal Transport Geometry \\
- Particles samplers \\
- Kernel methods for density estimation and classification \\
- Optimization methods for Reinforcement Learning }
\cventry{2015-2016}{Prophesee}{Paris, France}{}{}{
\textit{Computer Vision Research Engineer\\}
-Real-time multi-target tracking algorithms for Neuromorphic event-based cameras.\\
-Auto-calibration algorithms for event-based stereo cameras using structure from motion.}
\cventry{2015}{Ecole Normale Sup\'{e}rieure}{Paris, France}{}{}{
\textit{Visiting Student Researcher\\}
Advisor: Stéphane Mallat\\
Recurrent networks  for long-range dependencies in simple grammars using wavelets.}
\cventry{2014}{Princeton University}{Princeton, USA}{}{}{
\textit{Visiting Student Researcher\\}
Advisor: René Carmona\\
Mean field games with a dependence on the distribution of the  control: an Existence and Uniqueness result.}

\section{Honors and Awards}
\cvitem{2020}{Long Oral presentation at ICML 2021 for paper [3].  ~14\% of accepted papers, ~3\% of submitted papers).} 
\cvitem{2020}{Spotlight presentation at ICLR 2020 for paper [3]. ~2\% of submitted papers.}                             
\cvitem{2018}{Best Poster Award at MSR AI Summer School 2018 for paper [5]. Cambridge.}                         
\cvitem{2016-present}{Fully Funded PhD Scholarship. Awarded  by the Gatsby Computational Neuroscience Unit.}
\cvitem{2014}{Award of the Financial Risk Chair of École Polytechnique for research on Mean Field Games.}
\cvitem{2011--2014}{Fully Funded Masters Scholarship. Awarded by the French Government Eiffel Excellence Scholarship.}
\cvitem{2011--2014}{Fully Funded Undergraduate Scholarship. Awarded by the French Government Eiffel Excellence Scholarship.}
\cvitem{2009--2011}{National Moroccan merit scholarship ‘ISTIHQAQ’. For the first 100 national scores of high school final examinations.}
\cvitem{2009}{Participation to the 50th International Mathematics Olympiad (IMO), Bremen, Germany.}


\section{Research Activities}
%\phantom{\cite{Arbel:2018,Arbel:2018a,Arbel:2021a,Arbel:2019,Arbel:2020}}
\cventry{2018-present}{Reviewer}{}{}{}{-International conferences: NeurIPS (2018, 2019,2020), ICLR (2019,2020),ICML (2021)\\
-International journals: JMLR }
\cventry{2017-2019}{Research Seminars}{UCL}{}{}{
-Co-organized the DeepMind/CSML weekly research seminars in Machine Learning with invited speakers from the UK and Europe.\\  
-Co-organized pre-conference oral presentations events for selected accepted papers.\\
-Managed an annual budget of 5000 \pounds. 
}
\cventry{2017-present}{Member of the Machine Learning Journal Club}{UCL}{}{}{Weekly held journal club on various topics in Machine Learning and Statistics.}
\section{Publications}
\cvitem{2022}{\textbf{Michael Arbel}, and Julien Mairal. Amortized Implicit Differentiation for Stochastic Bilevel Optimization.
International Conference on Learning Representations (ICLR)}
\cvitem{2022}{Ted Moskovitz, \textbf{Michael Arbel}, Jack Parker-Holder, and Aldo Pacchiano. Towards an Understanding of Default Policies in Multitask Policy Optimization. 
International Conference on Artificial Intelligence and Statistics (AISTATS)}
\cvitem{2021}{\textbf{Michael Arbel*},  Alexander GDG Matthews*, Arnaud Doucet. Annealed Flow Transport Monte Carlo. International Conference on Machine Learning (ICML).  *equal contribution.}
\cvitem{2021}{Ted Moskovitz, Jack Parker-Holder, Aldo Pacchiano, \textbf{ Michael Arbel}, Jordan, Michael I. Tactical Optimism and Pessimism for Deep Reinforcement Learning.  \textit{Adv. Neural Information Processing Systems (NeurIPS)}}
\cvitem{2020}{\textbf{Michael Arbel*},  Ted Moskovitz*, Ferenc Huszar,  Arthur Gretton. Efficient Wasserstein Natural Gradients for Reinforcement Learning.  \textit{International Conference on Learning Representations (ICLR).}  *equal contribution.}
\cvitem{2020}{\textbf{Michael Arbel},  Liang Zhou, Arthur Gretton. Generalized Energy Based Models.  \textit{International Conference on Learning Representations (ICLR)}.}
\cvitem{2020}{Samuel Cohen, \textbf{Michael Arbel}, Marc Peter Deisenroth. Estimating Barycenters of Measures in High Dimensions.  \textit{arXiv preprint }.}
\cvitem{2020}{Louis Thiry, \textbf{Michael Arbel}, Eugene Belilovsky, Edouard Oyallon. The Unreasonable Effectiveness of Patches in Deep Convolutional Kernels Methods.  \textit{International Conference on Learning Representations (ICLR)}.}
\cvitem{2020}{Anna Korba, Adil Salim, \textbf{Michael Arbel}, Giulia Luise, Arthur Gretton. A Non-Asymptotic Analysis for Stein Variational Gradient Descent.\textit{ NeurIPS 2020}.}
\cvitem{2020}{Tolga Birdal, \textbf{Michael Arbel}, Umut Simsekli, Leonidas Guibas. Synchronizing Probability Measures on Rotations via Optimal Transport.  \textit{CVPR 2020}.}
\cvitem{2020}{\textbf{Michael Arbel}, Arthur Gretton, Wuchen Li, Guido Montufar. Kernelized Wasserstein Natural Gradient.  \textit{ICLR 2020}.}
\cvitem{2019}{\textbf{Michael Arbel}, Anna Korba, Adil Salim, Arthur Gretton. Maximum Mean Discrepancy Gradient Flow.  \textit{NeurIPS 2019}. }

\cvitem{2018}{\textbf{Michael Arbel*}, Dougal J. Sutherland*, Mikołaj Bińkowski, Arthur Gretton. On gradient regularizers for MMD GANs.  \textit{NeurIPS 2018}. *equal contribution.}

\cvitem{2018}{Mikołaj Bińkowski, Dougal J. Sutherland, \textbf{Michael Arbel}, Arthur Gretton. Demystifying MMD GANs. \textit{ICLR 2018}. }

\cvitem{2018}{Dougal J. Sutherland, Heiko Strathmann, \textbf{Michael Arbel}, Arthur Gretton. Efficient and principled score estimation. \textit{AISTATS 2018}. }
\cvitem{2018}{\textbf{Michael Arbel}, Arthur Gretton. Kernel Conditional Exponential Family. \textit{AISTATS 2018}. }
 
\section{Teaching Experience}
\cventry{2019}{Machine Learning Summer School }{London, UK}{}{}
{\textit{Teaching assistant} for the \textbf{Optimization} Tutorial.} 
\cventry{2017}{Gatsby Computational Neuroscience Unit - UCL }{London,  UK}{}{}
{\textit{Teaching assistant} for Probabilistic and Unsupervised Learning.\\
\textit{Teaching assistant} for Advanced Topics in Machine Learning.} 
 



\section{Invited Talks}

\cvitem{2020}{Kernelized Wasserstein Natural Gradient. Workshop on Functional Inference and Machine Intelligence, EURECOM (Sophia Antipolis, France).}

\cvitem{2020}{Maximum Mean Discrepancy Gradient flow.. The Alan Turing Institute (London, UK)..}

\cvitem{2020}{Wasserstein Natural Gradient: a kernel perspective. Department of Statistics, University of Oxford (Oxford, UK).}

\cvitem{2019}{Kernelized Wasserstein Natural Gradient. The Alan Turing Institute (London, UK)..}

\cvitem{2019}{Maximum Mean Discrepancy Gradient flow.
Amazon Research Days (Berlin, Germany).}

\cvitem{2019}{MMD Gradient flow. Workshop on Recent developments in kernel methods, 2019, UCL (London, UK).}

\cvitem{2019}{Kernel Distances for Deep Generative Models. Deep Learning Theory Kickoff Meeting 2019, MPI (Leipzig, Germany).}

\cvitem{2018}{On Gradient Regularizers for MMD-GANs. Cambridge-Tübingen workshop 2018 (Tenerife, Spain).}

\cvitem{2018}{Gradient Regularizers for MMD-GANs. Google Developer Group Reading and Thames Valley (Reading, UK).}


\section{Software}
\cvitem{2020}{Pytorch implementation of  Generalized Energy Based Models [1]. BSD 3-Clause License}
\cvitem{2019}{Pytorch implementation of the Measure synchronization on quaternion manifolds based on paper [5]. BSD 3-Clause License}
\cvitem{2019}{KWNG: Pytorch implementation of the optimizer based on paper [6]. BSD 3-Clause License}
\cvitem{2019}{MMDflow: Pytorch implementation of the noise-injection algorithm based on paper [7]. BSD 3-Clause License}
\cvitem{2018}{SMMD-GAN: Tensoflow implementation of scaled MMD-GAN based on paper [8]. BSD 3-Clause License}
\cvitem{2018}{KCEF: Python implementation the conditional density estimator based on paper [11]. BSD 3-Clause License}
%----------------------------------------------------------------------------------------
%	COMPUTER SKILLS SECTION
%----------------------------------------------------------------------------------------
\section{Languages}
\cvitem{Natural}{English (full proficiency), French (native).}
\cvitem{Programming}{Python, C++, Pytorch and Tensorflow.}





%----------------------------------------------------------------------------------------
%	COMMUNICATION SKILLS SECTION
%----------------------------------------------------------------------------------------
%----------------------------------------------------------------------------------------
%	LANGUAGES SECTION
%----------------------------------------------------------------------------------------

%----------------------------------------------------------------------------------------
%	INTERESTS SECTION
%----------------------------------------------------------------------------------------

%----------------------------------------------------------------------------------------

\end{document}